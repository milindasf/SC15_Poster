\documentclass[a4paper,10pt]{article}
%\documentclass[a4paper,10pt]{scrartcl}

\usepackage[utf8]{inputenc}

\title{SC 15 Five minute Presentation}
\author{}
\date{}

\pdfinfo{%
  /Title    ()
  /Author   ()
  /Creator  ()
  /Producer ()
  /Subject  ()
  /Keywords ()
}

\begin{document}
\maketitle

\begin{itemize}

\item Traditional partitioning schemes focus on dividing the total work equally among the processors (ideal/perfect load balancing) which will introduce communication cost imbalances between nodes.
\item These communication imbalances can decreases the overall performance in program execution. 
\item How octrees can be used to store hexahedral meshes.
\item Nearest common ancestor of two octants.
\item What is a Space filling curve and how it can be used in data partitioning.
\item Comparison between graph partitioning and Space filling curve based partitioning.
\item Contribution I: We are focus of developing partitioning scheme which introduces some flexibility in load balancing, and use allowed flexibility in load balancing to reduce the communication cost imbalance and overall
communication cost.
\item Comparison of flexible partitioning results of Hilbert and Morton Curve. 
\item Contribution II: We develop a new algorithm to calculate the Hilbert ordering based on the NCA approach which is almost efficient as Morton curve and 9 times faster than the traditional recursive approach of calculating
Hilbert curve.
\end{itemize}


\end{document}
