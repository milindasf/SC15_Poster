\documentclass[a4paper,10pt]{article}
%\documentclass[a4paper,10pt]{scrartcl}

\usepackage[utf8]{inputenc}

\title{SC 15 One minute Presentation}
\author{}
\date{}

\pdfinfo{%
  /Title    ()
  /Author   ()
  /Creator  ()
  /Producer ()
  /Subject  ()
  /Keywords ()
}

\begin{document}
\maketitle

\begin{itemize}
 \item Traditional partitioning schemes focus on dividing the total work equally among the processors (ideal/perfect load balancing) which will introduce communication cost imbalances between nodes. 
 \item These communication imbalances can decreases the overall performance in program execution. 
 \item We introduce new SFC based partitioning scheme which allows some flexibility in partitioning. 
 \item Flexibility allowed in partitioning can be use to reduce communication imbalances between nodes, leading to increased performance.
 \item Our results show that, instead of traditionally used Morton curve based partitioning, using Hilbert curve based partitioning reacts better with the flexible partitioning, providing
 better locality properties compared to Morton curve. 
 \item Since computation of Hilbert ordering is not trivial compared to Morton ordering we present a new algorithm based on Nearest Common Ancestor (NCA) for Hilbert ordering which is almost efficient as Morton ordering and 9
 times faster than traditional Hilbert ordering calculation based on recursive approach. 
\end{itemize}




\end{document}
