%The first is the standard SFC-based partitioning where each process gets assigned equal work and relies on the locality of SFCs for minimizing the communication costs. The second approach, allows for a user specified flexibility in the work asigned with the goal of further minimizing the communication costs. 
  
% In this paper we present a Hilbert curve based flexible dynamic partitioning scheme for adaptive scientific computations. Unlike most partitioning schemes our approach is not focused on perfect load balancing, which can increase communication costs. Instead of traditional approach, we introduce some flexibility (slack) to the load at each node, reducing the communication costs further. Our results show that load balancing with some flexibility allowed is [xx]\% efficient than the traditional partitioning approaches and it can reduce the overall computation time by [xx]\%.We use Hilbert curve instead of commonly used Morton curve in order to preserve the geometric locality more efficiently. Since, current implementations of Hilbert curve computations are expensive, we propose a new algorithm that works for all SFC using the property of the NCA. Our results show that the NCA based Hilbert ordering computation is [XX] times faster than the traditional recursive approach. 

